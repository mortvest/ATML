\documentclass[a4paper]{article}
\usepackage{amsmath}
\usepackage{amsfonts}
\usepackage{amsthm}
\usepackage{amssymb}
\usepackage[english]{babel}
\usepackage{float}
\usepackage{graphicx}
\usepackage{grffile}
\usepackage{hyperref}
\usepackage[utf8]{inputenc}
\usepackage{listings}
\usepackage{xcolor}
%% \usepackage{subfigure}
\usepackage{pdfpages}
\usepackage{graphicx}
\usepackage{subcaption}
\usepackage{stmaryrd}
\usepackage{a4wide}

\lstset{
  frame=tb,
  language=Python,
  aboveskip=3mm,
  belowskip=3mm,
  showstringspaces=false,
  formfeed=newpage,
  tabsize=4,
  comment=[l]{\#},
  breaklines=true,
  basicstyle=\small
}
\renewcommand{\thesubsubsection}{\alph{subsubsection}}
\newcommand{\prob}[1]{\mathbb{P}\left(#1\right)}
\newcommand{\expect}[1]{\mathbb{E}\left(#1\right)}
\newcommand{\avg}[1]{\sum_{i=1}^{#1}X_i}
\newcommand{\dotpr}[2]{\langle #1,\; #2 \rangle}
\newcommand{\norm}[1]{\left\lVert#1\right\rVert}
\newcommand*{\QEDA}{\hfill\ensuremath{\blacksquare}}%
\newcommand*{\bs}[1]{\boldsymbol{#1}}


\title{\vspace{-5cm}ATML Home Assignment 6}
\author{Dmitry Serykh (qwl888)}

\begin{document}
\maketitle
\section{CMA-ES for Reinforcement Learning}
\label{sec:1}

\section{Tighter analysis of the Hedge algorithm}
\label{sec:2}
\subsection{}
\label{subsec:21}
Let us recall Lemma 2.6 from the lecture notes:\\\\
Let $X$ be a random variable, such that $X \in [a,b]$, then for any $\lambda \in \mathbb{R}$:
\begin{align}
  \label{hoeff}
  \mathbb{E}\left[e^{\lambda X}\right] \leq e^{\lambda \mathbb{E}[X]+\frac{\lambda^{2}(b-a)^{2}}{8}}
\end{align}
We follow the proof of Lemma 5.2 from the lecture notes, until we reach (5.2):
\begin{align*}
  \frac{W_t}{W_{t-1}} &= \sum_a e^{-\eta X_t^a}p_t(a)\\
  &= \mathbb{E}_{p_t}[e^{-\eta X_t^a}]
\end{align*}
Then we can apply the Hoeffding's lemma, where the values of $X_t^{a} \in [0,1]$,
and $\lambda = -\eta$. Hence we get:
\begin{align*}
  \mathbb{E}_{p_t}[e^{-\eta X_t^a}] &\leq e^{-\eta \expect{X_t^a} + \frac{\eta^2}{8}}\\
  &=e^{-\eta \sum_a X_t^a p_t(a) + \frac{\eta^2}{8}}
\end{align*}
Then:
\begin{align*}
  \frac{W_{T}}{W_{0}}=\frac{W_{1}}{W_{0}} \times \frac{W_{2}}{W_{1}} \times
  \cdots \times \frac{W_{T}}{W_{T-1}} &\leq
  e^{-\eta \sum_{t=1}^{T} \sum_{a} X_{t}^{a} p_{t}(a)+ \sum_{t=1}^{T}\frac{\eta^{2}}{8}}
\end{align*}
Then, we combine this result with the other inequality from the proof in the lecture
notes and apply the logarithm:
\[
-\eta \min _{a} L_{T}(a)-\ln K \leq
-\eta \sum_{t=1}^{T} \sum_{a} X_{t}^{a} p_{t}(a)+
\frac{\eta^{2}}{8} T
\]
Then, we divide by eta and change sides:
\[
\sum_{t=1}^{T} \sum_{a} X_{t}^{a} p_{t}(a)-\min _{a} L_{T}(a)
\leq
\frac{\ln K}{\eta}+\frac{\eta}{8}T
\]
And we get:
\[
\mathbb{E}\left[R_{T}\right] \leq \frac{\ln K}{\eta}+\frac{\eta}{8} T
\]

\subsection{}
\label{subsec:22}
Then we find the value of $\eta$ that minimizes the bound by first finding the
first and second derivatives of the right side:
\begin{align*}
  \frac{d}{d\eta} \left[\frac{\ln{K}}{\eta} + \frac{\eta}{8}T\right]&=
  \frac{T}{8} - \frac{\ln{K}}{\eta^2}\\
  \frac{d}{d\eta} \left[\frac{T}{8} - \frac{\ln{K}}{\eta^2} \right]
  &= \frac{2\ln{K}}{\eta^3} \geq 0 \tag{since $\eta > 0$ and $K>1$}
  %% \frac{2\ln{K)}{\eta^3}
\end{align*}
We set the first derivative equal to zero and solve for $\eta$:
\begin{align*}
\frac{T}{8} - \frac{\ln{K}}{\eta^2} &= 0\\
\ln{K} &= \frac{T}{8}\eta^2\\
\eta &= \sqrt{\frac{8\ln{K}}{T}}
\end{align*}
Since second derivative is positive, the extremal point is a minimum.
\subsection{}
\label{subsec:23}
By setting $\eta$ to the value that we previously found in the bound, we get:
\begin{align*}
  \mathbb{E}\left[R_{T}\right] &\leq
  \frac{\ln K}{\sqrt{\frac{8\ln{K}}{T}}}+\frac{\sqrt{\frac{8\ln{K}}{T}}}{8} T\\
  &= \frac{8\ln{K} + 8\ln{K}} {8\sqrt{\frac{8\ln{K}}{T}}}\\
  &= \frac{2\ln{K}}{\sqrt{\frac{8\ln{K}}{T}}}\\
  &= \sqrt{T} \frac{\ln{K}}{\sqrt{2\ln{K}}}\\
  &= \sqrt{\frac{1}{2}T} \frac{\ln{K}}{\sqrt{\ln{K}}}\\
  &= \sqrt{\frac{1}{2}T \ln{K}}
\end{align*}
That is exactly the bound that we should obtain.

\section{The doubling trick}
\label{sec:3}
\subsection{}
\label{subsec:31}
We will first prove a lemma that would be useful later. In the prove we will use
the fact that we are allowed to square both sides of inequality if they are positive.
Let $m\geq0$, then:
\begin{align*}
  \sqrt{2}^m -1 \leq \sqrt{2^m -1}\\
  \sqrt{2^m} -1 \leq \sqrt{2^m -1}\\
  (\sqrt{2}^m -1)^2 \leq 2^m -1\\
  2^m - 2\sqrt{2^m} + 1 \leq 2^m - 1\\
  -2\sqrt{2^m} \leq  -2\\
  \sqrt{2^m} \geq  1\\
  2^m \geq  1
\end{align*}
This is clearly true, since $m\geq 0$.\\\\
With $\eta_{m}=\sqrt{\frac{8 \ln N}{2^{m}}}$, the expected regret of Hedge
within the period $(2^m,...,2^{m+1}-1)$ is bounded by $\sqrt{\frac{1}{2} 2^m\ln{K}}$.
Then for any $T=2^m - 1$, within the period $(1,...,T)$, we have:
\begin{align*}
  \expect{R_T} &\leq \sum_{j=0}^{j=m-1} \sqrt{\frac{1}{2}2^m\ln{K}}\\
  &=\sqrt{\frac{1}{2}\ln{K}}\sum_{j=0}^{j=m-1} \sqrt{2}^m\\
  &=\sqrt{\frac{1}{2}\ln{K}} \cdot \frac{1-\sqrt{2}^m}{1-\sqrt{2}} \tag{sum o
    geom. series}\\
  &=\sqrt{\frac{1}{2}\ln{K}} \left( \frac{1}{1-\sqrt{2}} - \frac{\sqrt{2}^m}{1-\sqrt{2}} \right)\\
  &=\sqrt{\frac{1}{2}\ln{K}} \left(
  \frac{1}{\sqrt{2}-1} \cdot  (\sqrt{2}^m -1)
  \right)\\
  &\leq \sqrt{\frac{1}{2}\ln{K}} \cdot \frac{1}{\sqrt{2} - 1} \sqrt{2^m - 1}
  \tag{by lemma, since $m\geq0$}\\
  &=\frac{1}{\sqrt{2} - 1} \sqrt{\frac{1}{2}T\ln{K}}
\end{align*}
\QEDA

\subsection{}
\label{subsec:32}
Any time period can be described as $T = 2^m + T_C$, for some $m\geq0$ and
$0 \leq T_C < 2^m$. \\
Let $T' = 2^{m+1}-1$, then $T \leq T'$ and since regret can not decrease with time,
$\expect{R_T} \leq \expect{R_{T'}}$.
Then, we can use the results from previous subsection and get:
\begin{align*}
  \mathbb{E}\left[R_{T}\right]&\leq \mathbb{E}\left[R_{T'}\right]\\
  &\leq\frac{1}{\sqrt{2}-1} \sqrt{\frac{1}{2} T' \ln K}\\
  &= \frac{1}{\sqrt{2}-1} \sqrt{\frac{1}{2} (2^{m+1}-1) \ln K}\\
  &\leq\frac{1}{\sqrt{2}-1} \sqrt{\frac{1}{2} (2^{m+1}+2T_C) \ln K} \tag{since $T_C \geq0$}\\
  &\leq\frac{\sqrt{2}}{\sqrt{2}-1} \sqrt{\frac{1}{2} (2^m+T_C) \ln K}\\
  &\leq\frac{\sqrt{2}}{\sqrt{2}-1} \sqrt{\frac{1}{2} T \ln K}
\end{align*}
\QEDA


\section{Empirical evaluation of algorithms for adversarial environments}
\label{sec:4}
It is relatively easy to show that an algorithm is bad in the adversarial
  environment. I made something very similar in Subsection \ref{subsec:52}.
  I showed that UCB1 can exhibit linear regret for a specially constructed
  sequence of rewards/losses. However, it says nothing about the difficulty of
  constructing such sequence, which could be very challenging.\\\\
However, it is clear that it is almost impossible to experimentally show that an
  algorithm is good in an adversarial environment. I showed in Subsection
  \ref{subsec:52} that EXP3 performs significantly better for a specific
  sequence of rewards that was designed to break UCB1. However, this does not
  mean that EXP3 performs good for all such sequences. The quality of such
  algorithms can only be shown with a formal proof.

\section{Empirical comparison of UCB1 and EXP3 algorithms}
\label{sec:5}
\subsection{Evaluation}
\label{subsec:51}
I implemented UCB1 from the notes, the improved UCB (that uses
$\hat{\mu}_{t-1}(a)+\sqrt{\frac{\ln t}{N_{t-1}(a)}}$) and EXP3 as defined in the
lecture notes. The source code for my implementation can be seen in file
\texttt{e5.py} in the \textbf{handin.zip}. I used the required settings and the
average results of 10 repetitions. Furthermore, I used the suggested formula for
the empirical pseudo-regret:
\[
\bar{R}_{t}^{\mathrm{emp}}=\sum_{s=1}^{t} N_{t}(a) \Delta(a)
\]
The results of the comparison can can be seen on 12 plots on Figure \ref{plt1} and \ref{plt2}.
UCB1 outperforms EXP3 for $K<8$, but EXP3 is better for some cases
when $K=8$ and for all cases where $K=16$.

\subsection{Breaking UCB1}
\label{subsec:52}
In order to break the UCB1 in the adversarial environment, I designed a
deterministic sequence using following method:
\begin{itemize}
\item The adversary has the knowledge of the used algorithm, hence I made use of
  a modified UCB1 algorithm in order to generate a sequence of rewards that
  would maximize the regret. 
\item I start by giving rewards of 1 to the first hand and 0 to others during the
  initialization phase. That way, I can keep track of  which hand is considered to be
  the best by the algorithm.
\item Then, I run and iteration of UCB1, and when it chooses the best hand to
  play, I set the reward for that hand equal to 0.
\item I repeat the process until the number of iterations reach $T$ 
\end{itemize}
I executed UCB1 and EXP3 on my sequence for $K=2$ and created a plot for the
regret (Figure \ref{plt3}). I used deterministic tie-breaking and used a
following formula for the regret:
\[
R_{T}=
\max _{a} \sum_{t=1}^{T} r_{t}^{a}-
\sum_{t=1}^{T} r_{t}^{A_{t}}
\]
It can be clearly seen on the plot that UCB1 exhibits a linear regret, while
EXP3 is sublinear as proven during the last lecture.
\begin{figure}
  \centering
  \begin{subfigure}[b]{0.49\textwidth}
    \centering
    \includegraphics[width=\textwidth]{code/plt_k2_mu025}
  \end{subfigure}
  \begin{subfigure}[b]{0.49\textwidth}
    \centering
    \includegraphics[width=\textwidth]{code/plt_k2_mu0375}
  \end{subfigure}
  \begin{subfigure}[b]{0.49\textwidth}
    \centering
    \includegraphics[width=\textwidth]{code/plt_k2_mu04375}
  \end{subfigure}
  \begin{subfigure}[b]{0.49\textwidth}
    \centering
    \includegraphics[width=\textwidth]{code/plt_k4_mu025}
  \end{subfigure}
  \begin{subfigure}[b]{0.49\textwidth}
    \centering
    \includegraphics[width=\textwidth]{code/plt_k4_mu0375}
  \end{subfigure}
  \begin{subfigure}[b]{0.49\textwidth}
    \centering
    \includegraphics[width=\textwidth]{code/plt_k4_mu04375}
  \end{subfigure}
  \caption{Pseudo regret (average of 10 repetitions), where $\mu$ is the average
    regret of suboptimal hands}
  \label{plt1}
\end{figure}

\begin{figure}
  \centering
  \begin{subfigure}[b]{0.49\textwidth}
    \centering
    \includegraphics[width=\textwidth]{code/plt_k8_mu025}
  \end{subfigure}
  \begin{subfigure}[b]{0.49\textwidth}
    \centering
    \includegraphics[width=\textwidth]{code/plt_k8_mu0375}
  \end{subfigure}
  \begin{subfigure}[b]{0.49\textwidth}
    \centering
    \includegraphics[width=\textwidth]{code/plt_k8_mu04375}
  \end{subfigure}
  \begin{subfigure}[b]{0.49\textwidth}
    \centering
    \includegraphics[width=\textwidth]{code/plt_k16_mu025}
  \end{subfigure}
  \begin{subfigure}[b]{0.49\textwidth}
    \centering
    \includegraphics[width=\textwidth]{code/plt_k16_mu0375}
  \end{subfigure}
  \begin{subfigure}[b]{0.49\textwidth}
    \centering
    \includegraphics[width=\textwidth]{code/plt_k16_mu04375}
  \end{subfigure}
  \caption{Pseudo regret (average of 10 repetitions), where $\mu$ is the average
    regret of suboptimal hands, continued}
  \label{plt2}
\end{figure}
\begin{figure}
  \centering
    \includegraphics[width=0.7\textwidth]{code/plt_adv}
  \caption{Regret of UCB1 and EXP3 on the designed adversarial sequence}
  \label{plt3}
\end{figure}

\end{document}

%% \section{Empirical comparison of UCB1 and EXP3 algorithms}
%% \label{sec:5}
%% \begin{figure}
%%   \centering
%%   \begin{subfigure}[b]{0.49\textwidth}
%%     \centering
%%     \includegraphics[width=\textwidth]{code/plt_k2_mu025}
%%   \end{subfigure}
%%   \begin{subfigure}[b]{0.49\textwidth}
%%     \centering
%%     \includegraphics[width=\textwidth]{code/plt_k2_mu0375}
%%   \end{subfigure}
%%   \caption{Plots for Exercise 5}
%%   \label{plt5}
%% \end{figure}

%% \begin{lstlisting}[caption="Calculation of g"]
%% def calc_g(Xs, y, w):
%%     N = np.shape(Xs)[0]
%%     # use matrix X of xs instead of for-loop = much faster
%%     X = np.c_[Xs, np.ones(N)]
%%     num = y.T * X
%%     denum = 1 + np.exp(y * (w @ X.T))
%%     M = num.T/denum
%%     # return mean of each row
%%     return (-1 * np.mean(M, axis=1))
%% \end{lstlisting}
