\documentclass[a4paper]{article}
\usepackage{amsmath}
\usepackage{amsfonts}
\usepackage{amsthm}
\usepackage[english]{babel}
\usepackage{float}
\usepackage{graphicx}
\usepackage{hyperref}
\usepackage[utf8]{inputenc}
\usepackage{listings}
\usepackage{xcolor}
%% \usepackage{subfigure}
\usepackage{graphicx}
\usepackage{subcaption}
\usepackage{stmaryrd}

\usepackage{a4wide}
\usepackage{url}
%% \usepackage[left=2cm,top=2cm,bottom=1.5cm,right=2cm]{geometry}

\lstset{
  frame=tb,
  language=Python,
  aboveskip=3mm,
  belowskip=3mm,
  showstringspaces=false,
  formfeed=newpage,
  tabsize=4,
  comment=[l]{\#},
  breaklines=true,
  morekeywords={models, lambda, forms}
}


\newcommand{\prob}[1]{\mathbb{P}\left(#1\right)}
\newcommand{\expect}[1]{\mathbb{E}\left(#1\right)}
\newcommand{\avg}[1]{\sum_{i=1}^{#1}X_i}
%% \newcommand{\dotp}[2]{\langle #1 + #2 \rangle}
%% \newcommand{\dotp}[2]{\ensuremath{\frac{#1}{#2}}}
\newcommand{\dotpr}[2]{\langle #1,\; #2 \rangle}
\newcommand{\norm}[1]{\left\lVert#1\right\rVert}

\title{\vspace{-5cm} Assignment 3}
\author{Dmitry Serykh (qwl888)}

\begin{document}
\maketitle
%% \tableofcontents

%% \section*{Introduction}
%% \label{sec:intro}
%% This assignment was solved and written by myself, but the exercises were discussed
%% in a studygroup with following students:
%% \begin{itemize}
%% \item Frederik Thorøe
%% \item Thomas Kawiecki
%% \end{itemize}
\section{Grid World}
\section{Numerical comparison of kl inequality with its relaxations and with Hoeffding’s inequality}
\section{Occam’s razor with kl inequality}
\section{Refined Pinsker’s Lower Bound}


\section{The Importance of Independence}
Let us define Bernoulli random variables $X_1, X_2, ..., X_n$ with bias $0.5$.
Let the value of $X_i$ for $i>1$ be equal the value of $X_0$. Then the value of
$\sum_{i=1}^{n} X_{i}$ would always be either 0 or 1 and
$\mu = \mathbb{E}\left[X_{i}\right] = \frac{1}{2} \cdot 0 + \frac{1}{2} \cdot 1 = \frac{1}{2}$.\\
Hence, the average would not converge to $\mu$ and we have:
$$
\mathbb{P}\left\{\left|\mu-\frac{1}{n} \sum_{i=1}^{n} X_{i}\right| \geq \frac{1}{2}\right\}=1
$$

\end{document}

%% \begin{figure}
%%   \centering
%%   \begin{subfigure}[b]{\textwidth}
%%     \centering
%%     \includegraphics[scale=0.8]{handin/plt51}
%%     \caption{Classification of the training set}
%%   \end{subfigure}
%%   \begin{subfigure}[b]{\textwidth}
%%     \centering
%%     \includegraphics[scale=0.8]{handin/plt52}
%%     \caption{Classification of the test set}
%%   \end{subfigure}
%%   \caption{Exercise 5: Logistic Regression Applied to the Datasets}
%%   \label{plt5}
%% \end{figure}

%% \begin{lstlisting}[caption="Calculation of g"]
%% def calc_g(Xs, y, w):
%%     N = np.shape(Xs)[0]
%%     # use matrix X of xs instead of for-loop = much faster
%%     X = np.c_[Xs, np.ones(N)]
%%     num = y.T * X
%%     denum = 1 + np.exp(y * (w @ X.T))
%%     M = num.T/denum
%%     # return mean of each row
%%     return (-1 * np.mean(M, axis=1))
%% \end{lstlisting}
