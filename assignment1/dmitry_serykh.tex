\documentclass[a4paper]{article}
\usepackage{amsmath}
\usepackage{amsfonts}
\usepackage{amsthm}
\usepackage{amssymb}
\usepackage[english]{babel}
\usepackage{float}
\usepackage{graphicx}
\usepackage{hyperref}
\usepackage[utf8]{inputenc}
\usepackage{listings}
\usepackage{xcolor}
%% \usepackage{subfigure}
\usepackage{pdfpages}
\usepackage{graphicx}
\usepackage{subcaption}
\usepackage{stmaryrd}
\usepackage{a4wide}

\lstset{
  frame=tb,
  language=Python,
  aboveskip=3mm,
  belowskip=3mm,
  showstringspaces=false,
  formfeed=newpage,
  tabsize=4,
  comment=[l]{\#},
  breaklines=true,
  basicstyle=\small
}

\newcommand{\prob}[1]{\mathbb{P}\left(#1\right)}
\newcommand{\expect}[1]{\mathbb{E}\left(#1\right)}
\newcommand{\avg}[1]{\sum_{i=1}^{#1}X_i}
%% \newcommand{\dotp}[2]{\langle #1 + #2 \rangle}
%% \newcommand{\dotp}[2]{\ensuremath{\frac{#1}{#2}}}
\newcommand{\dotpr}[2]{\langle #1,\; #2 \rangle}
\newcommand{\norm}[1]{\left\lVert#1\right\rVert}
\newcommand*{\QEDA}{\hfill\ensuremath{\blacksquare}}%

\title{\vspace{-5cm} Assignment 1}
\author{Dmitry Serykh (qwl888)}

\begin{document}
\maketitle

\section{Grid World}
\label{sec:1}
In this exercise and this assignment in general, I exclusively used python (with
\texttt{numpy}, \texttt{matplotlib} and other relevant libraries) for all
implementation.
\subsection{}
\label{subsec:11}
In order to calculate the value function $V^{rand}$ of the random policy, I
implement a version of the iterative policy evaluation algorithm from the slide
31 in the reinforcement learning lecture slides:
$$
  V(s) \leftarrow \sum_{a} \pi(s, a)
  \sum_{s^{\prime}} P_{s s^{\prime}}^{a}\left[R_{s s^{\prime}}^{a}+\gamma
    V(s^{\prime})\right]
  $$
An important observation is that all rewards and
transitions in this setup are deterministic, hence I can simplify the expression in the update
stage of the algorithm ($P_{s s^{\prime}}^{a} = 1$ for the transition state and
0 otherwise):
\begin{align*}
  &\sum_{a} \pi(s, a) \sum_{s^{\prime}} P_{s s^{\prime}}^{a}\left[R_{s s^{\prime}}^{a}+\gamma
    V(s^{\prime})\right] = 
  \sum_{a} \pi(s, a)
  \left[R_{s s^{\prime}}^{a} + V(s^{\prime})\right]
\end{align*}
where $s^{\prime}$ is the state after the transition. $s^{\prime}$ is found
by a simple lookup in the transitional mapping. I implemented both mappings and policy
using $12\times 4$ matrixes, and my implementation can be found in file \texttt{e1.py} in the
\textbf{handin.zip}. The resulting value of $V^{rand}$ (with threshold 0.1) is:
\begin{verbatim}
[[-202.461 -200.126 -202.888]
 [-201.185 -186.308 -201.934]
 [-191.192 -168.74  -187.252]
 [-199.91   -86.331    0.   ]]
\end{verbatim}

\subsection{}
\label{subsec:12}
For this sub-task, I implemented the policy iteration algorithm
from the slide 37. The state transitions are still deterministic, hence I can
apply similar simplifications from the previous sub-task. Furthermore, I reuse
my implementation for the policy evaluation stage. \\\\
I will then highlight the calculation of \emph{argmax}, which I find
efficiently by multiplying a $4 \times 4$ identity matrix with the row in the
reward matrix and the value function of the target states. I then find the index
of the largest value in the result and return that row of the identity as the
action of that state.
A relevant code snippet can be seen on Listing \ref{lst1}.

\begin{lstlisting}[caption="Calculation of argmax\_a", label=lst1]
def find_best_action(s):
    diag_mat = np.diag(np.ones(4))
    diag_mult = diag_mat * (self.rew_mat[s] + self.Vs[self.trans_mat[s]])
    arr = np.diag(diag_mult)
    max_ind = np.argmax(arr)
    res_policy = diag_mat[max_ind]
    return res_policy
\end{lstlisting}
I initialize the algorithm with a random policy $V=0$ and use the same value of
$\theta$ as in the first sub-task. My implementation can be found in file
\texttt{e1.py} in \textbf{handin.zip} and the resulting value of $V^{*}$ is:
\begin{verbatim}
[[-10.  -9. -10.]
 [ -9.  -3. -11.]
 [ -3.  -2.  -3.]
 [ -9.  -1.   0.]]
\end{verbatim}


\section{Numerical comparison of kl inequality with its relaxations and with Hoeffding’s inequality}
\subsection{}
\label{subsec:21}
The four bounds on $p$ are in a form ``with probability greater than $1 - \delta$'' and can
be explicitly written as:
\begin{enumerate}
\item $$p \leq \hat{p}_n+\sqrt{\frac{\ln \frac{1}{\delta}}{2 n}}$$
\item
  \begin{align*}
  p \leq \mathrm{kl}^{-1^{+}}\left(\hat{p}_{n}, z\right)
  \end{align*}
  where the value of $z$ is taken from the inequality (2.11) in the lecture notes:
  $$ \mathrm{kl}(\hat{p} \| p) \leq \frac{\ln \frac{n+1}{\delta}}{n} $$
\item Pinskers inequality for the binary kl-divergence is given by:
  $$|p-\hat{p}_n| \leq \sqrt{\frac{\ln \frac{n+1}{\delta}}{2 n}}$$
  I only need the positive part, hence it can be rewritten:
  $$p \leq \hat{p}_n + \sqrt{\frac{\ln \frac{n+1}{\delta}}{2 n}}$$
\item $$ p \leq \hat{p}_n+\sqrt{\frac{2 \hat{p}_n \ln \frac{n+1}{\delta}}{n}}+\frac{2
  \ln \frac{n+1}{\delta}}{n}$$
\end{enumerate}

\subsection{}
\label{subsec:22}
The plot for the given values can be seen on Figure \ref{plt1}.

\subsection{}
\label{subsec:23}
The zoomed-in plot for the given values can also be seen on Figure \ref{plt1}.

\subsection{}
\label{subsec:24}
I implemented the ``lower inverse'' of kl by modifying the implementation of the
``upper inverse'', which was based on the provided solution in
python.
The plot for the given values can be seen on Figure \ref{plt2}.

\subsection{}
\label{subsec:25}
As discussed during the lecture, kl-inequality performs better for the smaller
values of $\hat{h}_n$ (0 to 0.08) and for the ones close to 1 (0.92 to 1).
Hoeffding binds tighter for other values. \\
Hoeffding's bound performs strictly better than Pinsker's relaxation.
Furthermore, the refined Pinsker's relaxation binds tighter for (0 to 0.03).
Generally, the difference could be significant when the prediction loss is
extremely low.

\section{Occam’s razor with kl inequality}
I start by looking at inequality (2.11) from the lecture notes, where with
probability greater than $1-\delta$:
$$
\mathrm{kl}(\hat{p} \| p) \leq \frac{\ln \frac{n+1}{\delta}}{n}
$$
An alternative way of looking at the theorem is:
\begin{equation}\label{kl1}
\mathbb{P}(
\mathrm{kl}(\hat{p} \| p) \geq \frac{\ln \frac{n+1}{\delta}}{n}
) \leq \delta \tag{1}
\end{equation}
I want to prove that for $\delta \in(0,1)$, with probability greater than
$1-\delta$ for all $h \in \mathcal{H}$:
$$
\mathrm{kl}(\hat{L}(h, S) \| L(h)) \leq \frac{\ln \frac{n+1}{\pi(h) \delta}}{n}
$$
But instead, I will prove the equivalent statement by following a method similar
to the proof of the Occam's Razor bound with the Hoeffding's inequality in the
lecture slides:
$$
  \mathbb{P}(\exists h \in \mathcal{H} :
  \mathrm{kl}(\hat{L}(h) \| L(h)) \geq \frac{\ln \frac{n+1}{\pi(h)\delta}}{n})
  \leq \delta
$$
I prove it by looking at the left-hand side of the inequality:
\begin{align*}
  \mathbb{P}(\exists h \in \mathcal{H} :
  \mathrm{kl}(\hat{L}(h) \| L(h)) \geq \frac{\ln \frac{n+1}{\pi(h)\delta}}{n})
  &\leq
  \sum_{h \in \mathcal{H}}
  \mathbb{P} (\mathrm{kl}(\hat{L}(h) \| L(h)) \geq \frac{\ln \frac{n+1}{\pi(h)\delta}}{n})\\
  &\leq
  \sum_{h \in \mathcal{H}}\pi(h)\delta \\
  &\leq
  \delta
\end{align*}
The first inequality holds because of the union bound.\\
The second inequality holds because of (\ref{kl1}) and the fact that $\pi(h)$ is
independent from $S$. Otherwise the $kl$ inequality could not be used here, as Lemma 2.14
from lecture notes is proven for $\varepsilon$ being a scalar value. The
notion of $\pi$ being dependent on $S$, would make $\varepsilon$ a random
variable and break the proof.\\
The last inequality holds because
$\sum_{h \in \mathcal{H}} \pi(h) \leq 1$. \QEDA

\section{Refined Pinsker’s Lower Bound}
I need to prove that if $\mathrm{kl}(p \| q) \leq \varepsilon$ then
$q \geq p-\sqrt{2 p \varepsilon}$. \\
$0 \leq p,q \leq 1$, since both of them denote a bias of a Bernoulli variable.
Furthermore, $\varepsilon \geq 0$, because evaluating $\mathrm{kl}(p \| q)$ always yields
a non-negative result. I will then consider two following cases:
\begin{enumerate}
\item $p \geq q$\\
  I start by looking at Lemma 1.18 in the lecture notes:
  \[
  \mathrm{kl}(p \| q) \geq
  \frac{(p-q)^{2}}{2 \max \{p, q\}}+\frac{(p-q)^{2}}{2 \max \{(1-p),(1-q)\}}
  \]
  Since $\mathrm{kl}(p \| q) \leq \varepsilon$:
  \begin{align*}
    \varepsilon &\geq
    \frac{(p-q)^{2}}{2 \max \{p, q\}}+\frac{(p-q)^{2}}{2 \max \{(1-p),(1-q)\}}\\
    &\geq
    \frac{(p-q)^{2}}{2 \max \{p, q\}}
    \tag{since $\varepsilon,p,q \geq 0$}\\
    &=
    \frac{(p-q)^{2}}{2p}
      \tag{since $p \geq q$}
\end{align*}
  I then solve for $q$:
  \begin{align*}
    \varepsilon &\geq \frac{(p-q)^{2}}{2p} \\
    2p\varepsilon &\geq (p-q)^{2} \tag{since $p \geq 0$}\\
    \sqrt{2p\varepsilon} &\geq p-q \\
    q &\geq p-\sqrt{2 p \varepsilon} \tag{since $q \geq 0$}\\
  \end{align*}
    The penultimate step is valid, since both sides of the inequality are positive and
    square root is a monotonically increasing, non-negative function.
\item $q \geq p$:\\
  This case is trivial to see, as both $p, \varepsilon$ are non-negative.
  The same holds for the square root function, therefore
  $\sqrt{2 p \varepsilon} \geq 0$. Hence:
  \begin{align*}
    q \geq p-\sqrt{2 p \varepsilon} \tag{since $q \geq p$ in this case}
  \end{align*}
\end{enumerate}
\QEDA

\section{The Importance of Independence}
Let us define Bernoulli random variables $X_1, X_2, ..., X_n$ with bias $0.5$.
Let them also be dependent, such that the value of $X_i = X_0$ for $i>1$. Then the value of
$\sum_{i=1}^{n} X_{i}$ would always be either 0 or 1 and
$\mu = \mathbb{E}\left[X_{i}\right] = \frac{1}{2} \cdot 0 + \frac{1}{2} \cdot 1 = \frac{1}{2}$.\\
Hence, the average would not converge to $\mu$ and we have for all $n > 0$:
$$
\mathbb{P}\left\{\left|\mu-\frac{1}{n} \sum_{i=1}^{n} X_{i}\right| \geq \frac{1}{2}\right\}=1
$$

\begin{figure}
  \centering
  \begin{subfigure}[b]{\textwidth}
    \centering
    \includegraphics[scale=0.8]{code/plt21}
    \caption{0 to 1}
  \end{subfigure}
  \begin{subfigure}[b]{\textwidth}
    \centering
    \includegraphics[scale=0.8]{code/plt22}
    \caption{Zoomed}
  \end{subfigure}
  \caption{Numerical comparison of kl inequality and relaxations with Hoeffding’s inequality}
  \label{plt1}
\end{figure}

\begin{figure}
  \centering
  \includegraphics[scale=0.8]{code/plt23}
  \caption{Numerical comparison of kl lower bound with Hoeffding’s inequality}
  \label{plt2}
\end{figure}
\end{document}

%% \begin{figure}
%%   \centering
%%   \begin{subfigure}[b]{\textwidth}
%%     \centering
%%     \includegraphics[scale=0.8]{handin/plt51}
%%     \caption{Classification of the training set}
%%   \end{subfigure}
%%   \begin{subfigure}[b]{\textwidth}
%%     \centering
%%     \includegraphics[scale=0.8]{handin/plt52}
%%     \caption{Classification of the test set}
%%   \end{subfigure}
%%   \caption{Exercise 5: Logistic Regression Applied to the Datasets}
%%   \label{plt5}
%% \end{figure}

%% \begin{lstlisting}[caption="Calculation of g"]
%% def calc_g(Xs, y, w):
%%     N = np.shape(Xs)[0]
%%     # use matrix X of xs instead of for-loop = much faster
%%     X = np.c_[Xs, np.ones(N)]
%%     num = y.T * X
%%     denum = 1 + np.exp(y * (w @ X.T))
%%     M = num.T/denum
%%     # return mean of each row
%%     return (-1 * np.mean(M, axis=1))
%% \end{lstlisting}
