\documentclass[a4paper]{article}
\usepackage{amsmath}
\usepackage{amsfonts}
\usepackage{amsthm}
\usepackage{amssymb}
\usepackage[english]{babel}
\usepackage{float}
\usepackage{graphicx}
\usepackage{hyperref}
\usepackage[utf8]{inputenc}
\usepackage{listings}
\usepackage{xcolor}
%% \usepackage{subfigure}
\usepackage{pdfpages}
\usepackage{graphicx}
\usepackage{subcaption}
\usepackage{stmaryrd}
\usepackage{a4wide}

\lstset{
  frame=tb,
  language=Python,
  aboveskip=3mm,
  belowskip=3mm,
  showstringspaces=false,
  formfeed=newpage,
  tabsize=4,
  comment=[l]{\#},
  breaklines=true,
  basicstyle=\small
}

\newcommand{\prob}[1]{\mathbb{P}\left(#1\right)}
\newcommand{\expect}[1]{\mathbb{E}\left(#1\right)}
\newcommand{\avg}[1]{\sum_{i=1}^{#1}X_i}
%% \newcommand{\dotp}[2]{\langle #1 + #2 \rangle}
%% \newcommand{\dotp}[2]{\ensuremath{\frac{#1}{#2}}}
\newcommand{\dotpr}[2]{\langle #1,\; #2 \rangle}
\newcommand{\norm}[1]{\left\lVert#1\right\rVert}
\newcommand*{\QEDA}{\hfill\ensuremath{\blacksquare}}%

\title{\vspace{-5cm} Assignment 2}
\author{Dmitry Serykh (qwl888)}

\begin{document}
\maketitle
\section{Policy Evaluation}
\label{sec:1}
I need to prove that:
\[
\forall s \in S : \lim _{k \rightarrow \infty} V_{k}(s)=V^{\pi}(s)
\]
This is equivalent to proving following:
\[
\forall s \in S : \lim _{k \rightarrow \infty} \Delta_{k} = 0
\]
I start by looking at the definition of $\Delta_k$ and replace the values of
$V$ with the update expressions from the reinforcement learning slides. Then for
all $s \in S$, we have:
\begin{align*}
  \Delta_{k} &= \max _{s}\left|V_{k}(s)-V^{\pi}(s)\right|\\
  &= \max _{s}\left|
  \sum_{a} \pi(s, a) \sum_{s^{\prime}} P_{s s^{\prime}}^{a}\left[R_{s s^{\prime}}^{a}+\gamma
    V_{k-1}\left(s^{\prime}\right)\right] -
  \sum_{a} \pi(s, a) \sum_{s^{\prime}} P_{s s^{\prime}}^{a}\left[R_{s s^{\prime}}^{a}+\gamma
    V^{\pi}\left(s^{\prime}\right)\right] \right| \\
  &= \max _{s}\left| \sum_{a} \pi(s, a)
  \sum_{s^{\prime}} P_{s s^{\prime}}^{a}\left[R_{s s^{\prime}}^{a}+\gamma
    V_{k-1}\left(s^{\prime}\right)\right] - P_{s s^{\prime}}^{a}\left[R_{s s^{\prime}}^{a}+\gamma
    V^{\pi}\left(s^{\prime}\right)\right]\right| \\
  &= \max _{s}\left| \sum_{a} \pi(s, a)\sum_{s^{\prime}} P_{s s^{\prime}}^{a} \gamma
  \left[V_{k-1}\left(s^{\prime}\right) - V^{\pi}\left(s^{\prime}\right)\right]
  \right| \\
  &= \gamma \max _{s}\left| \sum_{a} \pi(s, a)\sum_{s^{\prime}} P_{s s^{\prime}}^{a}
  \left[V_{k-1}\left(s^{\prime}\right) - V^{\pi}\left(s^{\prime}\right)\right]
  \right| \\
  & \leq \gamma  \max _{s}\left| \sum_{a} \pi(s, a)\sum_{s^{\prime}} P_{s s^{\prime}}^{a}
  \cdot \max _{s~{\prime}} \left|
  V_{k-1}\left(s^{\prime}\right) - V^{\pi}\left(s^{\prime}\right) \right|
  \right| \\
  & = \gamma \max _{s~{\prime}} \left|
  V_{k-1}\left(s^{\prime}\right) - V^{\pi}\left(s^{\prime}\right) \right| \cdot
  \max _{s}\left| \sum_{a} \pi(s, a)\sum_{s'} P_{s s^{\prime}}^{a}
  \right| \\
  &= \gamma \cdot \Delta_{k-1} \cdot
  \max _{s}\left| \sum_{a} \pi(s, a)\sum_{s'} P_{s s^{\prime}}^{a}
  \right| \\
  &= \gamma \cdot \Delta_{k-1} 
\end{align*}
The penultimate equality holds because of the definition of $\Delta$, while
the last equality holds because both $\pi(s,a)$ and
$P_{s s^{\prime}}^{a}$ are probabilities that are summed for all values of
$a$ and $s'$. \\
After the transformation, I get the inequality:
\[
\forall s \in S : \Delta_k  \leq \gamma \cdot \Delta_{k-1}
\]
The values of both $\Delta_k$ and $\Delta_{k-1}$ are nonnegative, because
of the absolute value in the definition. Furthermore $0 < \gamma < 1$, hence:
\[
\forall s \in S : \lim _{k \rightarrow \infty} \Delta_{k} = 0
\]
\QEDA

\newpage
\section{Maze Example}
\label{sec:2}
\subsection{}
\label{subsec:21}
I start by looking at the expression on the slide 19 in the RL lecture slides that states:
\[
V^{\pi}(s)
=\sum_{a} \pi(s, a) \sum_{s^{\prime}} P_{s s^{\prime}}^{a}
\left[R_{s s^{\prime}}^{a}+\gamma V^{\pi}\left(s^{\prime}\right)\right]
\]
It can be deducted from the rightmost picture that the policy $\pi$ is
deterministic (the arrow shows the action in each state), hence
$\pi(s,a) = \{0,1\}$. Therefore:
\[
V^{\pi}(s) =\sum_{s^{\prime}} P_{s s^{\prime}}^{a}
\left[R_{s s^{\prime}}^{a}+\gamma V^{\pi}\left(s^{\prime}\right)\right]
\]
where $a$ is an action in the state, determined by $\pi$.\\
Moreover, similarly to the first exercise in Assignment 1, that all rewards and
transitions are deterministic. Hence, I can simplify the expression:
\begin{align}
  \label{V_pi}
  V^{\pi}(s) = \sum_{s^{\prime}} R_{s s^{\prime}}^{a}+\gamma V^{\pi}\left(s^{\prime}\right)
\end{align}
According to the assignment text, any action in the goal state lets me stay in
the goal state. Furthermore, staying in the goal state gives an immediate reward of 1,
hence $s^{\prime}$ can only equal $G$ when $s = G$. Therefore I can further simplify
the expression:
\begin{align*}
  V^{\pi}(G) &= R_{G G}^{a}+\gamma V^{\pi}\left(G\right) \\
  &= 1 + 0.9 \cdot V^{\pi}(G)
\end{align*}
I can solve for $V^{\pi}(G)$ and get:
\[
V^{\pi}(G) = 10
\]
\QEDA

\subsection{}
\label{subsec:21}
Let $G_{left}$ and $G_{below}$ represent the states to the left and below of
state $G$ respectively. I start by looking at (\ref{V_pi}):
\begin{align*}
  V^{\pi}(s) = \sum_{s^{\prime}} R_{s s^{\prime}}^{a}+\gamma V^{\pi}\left(s^{\prime}\right)
\end{align*}
For both $G_{left}$ and $G_{below}$, the only transition state is $G$ (since
$\pi$ is deterministic). Furthermore, the reward of transition from one state to
another, while not leaving the world is 0, and I know the value of $V^{\pi}(G)$, hence:
\begin{align*}
  V^{\pi}(G_{left}) &= R_{G G_{left}}^{a}+\gamma V^{\pi}(G) \\
  V^{\pi}(G_{left}) &= 0+0.9 \cdot 10 \\
  V^{\pi}(G_{left}) &= 9 
\end{align*}
Same logic can be applied to the calculation of $V^{\pi}(G)$, hence:
\[
  V^{\pi}(G_{below}) = V^{\pi}(G_{left}) = 9 \\
\]
\QEDA
\newpage

\section{PAC-Bayes vs. Occam}
\label{sec:3}
\subsection{}
\label{subsec:31}
I start by rewriting an expression:
\[
\ln \frac{n+1}{\pi(h) \delta} 
=\ln{\left(\frac{1}{\pi(h)} \cdot \frac{n+1}{\delta}\right)}
=\ln{\frac{1}{\pi(h)}} + \ln{\frac{n+1}{\delta}}
\]
That would be useful later. \\\\
I then use the LHS of Theorem 1 and apply Corollary 3.25. from the
lecture notes and get:
\begin{align*}
\mathrm{kl}\left(\mathbb{E}_{\rho}[\hat{L}(h, S)] \|
\mathbb{E}_{\rho}[L(h)]\right) &\leq
\mathbb{E}_{\rho}[\mathrm{kl}(\hat{L}(h, S)] \| L(h))]\\
&\leq \mathbb{E}_{\rho}\left[\frac{\ln \frac{n+1}{\pi(h) \delta}}{n}\right]\\
&= \mathbb{E}_{\rho}\left[\frac{\ln{\frac{1}{\pi(h)}} + \ln{\frac{n+1}{\delta}}}{n}\right] \\
&= \frac{\mathbb{E}_{\rho}\left[\ln{\frac{1}{\pi(h)}}\right] + \ln{\frac{n+1}{\delta}}}{n} \\
&= \frac{\sum_{h \in \mathcal{H}}\rho(h)\ln{\frac{1}{\pi(h)}} + \ln{\frac{n+1}{\delta}}}{n}
\end{align*}
The second inequality holds because of the results from the Home Assignment 1,
which holds with probability greater than $1-\delta$ for all $h \in \mathcal{H}$
simultaneously.The penultimate equality holds because neither $\ln{\frac{n+1}{\delta}}$ or $n$
depend on $h$. In the last equality, the definition of expected value is used.\\\\
Then with probability greater than $1-\delta$, for all distributions $\rho$
over $\mathcal{H}$ simultaneously:
\begin{align}
  \label{T1}
  \mathrm{kl}\left(\mathbb{E}_{\rho}[\hat{L}(h, S)] \|
  \mathbb{E}_{\rho}[L(h)]\right) \leq
  \frac{\sum_{h \in \mathcal{H}} \rho(h) \ln \frac{1}{\pi(h)}+\ln \frac{n+1}{\delta}}{n}
\end{align}
\QEDA
\subsection{}
\label{subsec:32}
Let us recall the definition of the relative entropy of Kullback-Leibler
divergence, where $p(x)$ and $q(x)$ are two probability distributions:
\[
\mathrm{KL}(p \| q)=
\sum_{x \in \mathcal{X}} p(x) \ln \frac{p(x)}{q(x)}
\]
I can insert the definition in the PAC-Bayes-kl inequality, that holds with probability greater
than $1-\delta$, for all distributions $\rho$ over $\mathcal{H}$ simultaneously:
\begin{align}
  \label{T2}
\mathrm{kl}\left(\mathbb{E}_{\rho}[\hat{L}(h, S)] \|
\mathbb{E}_{\rho}[L(h)]\right)
\leq \frac{\mathrm{KL}(\rho \| \pi)+\ln \frac{n+1}{\delta}}{n}
=\frac{\sum_{h \in \mathcal{H}} p(h) \ln \frac{p(h)}{q(h)}+\ln \frac{n+1}{\delta}}{n}
\end{align}
Therefore, the bound in (\ref{T2}) is at least as tight as the one in (\ref{T1}),
because the bounds differ by the factor $0 \leq p(h) \leq 1$ in the log
term inside the sum.
\newpage

\section{Nonnegativity of KL}
\label{sec:4}
\subsection{}
\label{subsec:41}
I need to prove that $\ln x \leq x-1$ for all $0<x<\infty$. I start by
differentiating both functions:
\begin{align*}
&\frac{d}{dx}\ln x = \frac{1}{x} \\
&\frac{d}{dx}x-1 = 1 
\end{align*}
for $0 < x < 1:$
\[
\frac{1}{x} > 1
\]
for $x > 1:$
\[
\frac{1}{x} < 1 
\]
$\ln x = x-1$ has a single solution for $x=1$, hence for values $x>1$:
$$\ln x < x - 1$$


\subsection{}
\label{subsec:42}
According to Definition 2.11 from the lecture notes and since
$\mathcal{X}$ is discrete:
\[
\mathrm{KL}(p \| q)=\mathbb{E}_{p}\left[\ln \frac{p(X)}{q(X)}\right]
= \sum_{x \in \mathcal{X}} p(x) \ln \frac{p(x)}{q(x)}
\]
First, I look at the case where there is a point $x$ for which
$q(x) = 0$ and $p(x) > 0$; and the case where $p(x) = 0$:
In the lecture notes, a convention is used where
$0 \ln \frac{0}{0}=0 \text { and } 0 \ln \frac{0}{q}=0 \text{ and } p \ln \frac{p}{0}=\infty$,
hence our inequality holds. \\\\
For the case where $p>0$ and  $q>0$, I follow the hint and will prove that
$-\mathrm{KL}(p \| q) \leq 0$:
\begin{align*}
  -\sum_{x \in \mathcal{X}} p(x) \ln \frac{p(x)}{q(x)}
  &=    -\sum_{x \in \mathcal{X}} p(x) \ln (p(x) - q(x)) \\
  &= \sum_{x \in \mathcal{X}} p(x) \ln (q(x) - p(x)) \\
  &= \sum_{x \in \mathcal{X}} p(x) \ln \frac{q(x)}{p(x)} \\
  &\leq \sum_{x \in \mathcal{X}} p(x)\left(\frac{q(x)}{p(x)}-1\right)
  \tag{by Subsection \ref{subsec:41}}\\
  &= \sum_{x \in \mathcal{X}} \left(q(x) - p(x)\right) \\
  &= \sum_{x \in \mathcal{X}} q(x) - \sum_{x \in \mathcal{X}} p(x)\\
  &= \sum_{x \in \mathcal{X}} q(x) - 1 \\
  &\leq 0
\end{align*}
The last inequality holds, because of the way we chose $q$.

\subsection{}
\label{subsec:43}
In my proof, $\mathrm{KL}(p \| q)=0$ would hold iff $p(x) = q(x)$



\end{document}

%% \begin{figure}
%%   \centering
%%   \begin{subfigure}[b]{\textwidth}
%%     \centering
%%     \includegraphics[scale=0.8]{handin/plt51}
%%     \caption{Classification of the training set}
%%   \end{subfigure}
%%   \begin{subfigure}[b]{\textwidth}
%%     \centering
%%     \includegraphics[scale=0.8]{handin/plt52}
%%     \caption{Classification of the test set}
%%   \end{subfigure}
%%   \caption{Exercise 5: Logistic Regression Applied to the Datasets}
%%   \label{plt5}
%% \end{figure}

%% \begin{lstlisting}[caption="Calculation of g"]
%% def calc_g(Xs, y, w):
%%     N = np.shape(Xs)[0]
%%     # use matrix X of xs instead of for-loop = much faster
%%     X = np.c_[Xs, np.ones(N)]
%%     num = y.T * X
%%     denum = 1 + np.exp(y * (w @ X.T))
%%     M = num.T/denum
%%     # return mean of each row
%%     return (-1 * np.mean(M, axis=1))
%% \end{lstlisting}
