\documentclass[a4paper]{article}
\usepackage{amsmath}
\usepackage{amsfonts}
\usepackage{amsthm}
\usepackage{amssymb}
\usepackage[english]{babel}
\usepackage{float}
\usepackage{graphicx}
\usepackage{grffile}
\usepackage{hyperref}
\usepackage[utf8]{inputenc}
\usepackage{listings}
\usepackage{xcolor}
%% \usepackage{subfigure}
\usepackage{pdfpages}
\usepackage{graphicx}
\usepackage{subcaption}
\usepackage{stmaryrd}
\usepackage{a4wide}

\lstset{
  frame=tb,
  language=Python,
  aboveskip=3mm,
  belowskip=3mm,
  showstringspaces=false,
  formfeed=newpage,
  tabsize=4,
  comment=[l]{\#},
  breaklines=true,
  basicstyle=\small
}
%% \renewcommand{\thesubsubsection}{\alph{subsubsection}}
\newcommand{\prob}[1]{\mathbb{P}\left(#1\right)}
\newcommand{\expect}[1]{\mathbb{E}\left(#1\right)}
\newcommand{\avg}[1]{\sum_{i=1}^{#1}X_i}
\newcommand{\dotpr}[2]{\langle #1,\; #2 \rangle}
\newcommand{\norm}[1]{\left\lVert#1\right\rVert}
\newcommand*{\QEDA}{\hfill\ensuremath{\blacksquare}}%
\newcommand*{\bs}[1]{\boldsymbol{#1}}


\title{\vspace{-5cm}ATML Home Assignment 7}
\author{Dmitry Serykh (qwl888)}

\begin{document}
\maketitle
\section{Policy Gradient Methods}
\label{sec:1}
\subsection*{Question 1.1}
\label{subsec:11}
I know from the assignment text that:
\[
\pi(s, a)=\frac{e^{\boldsymbol{\theta}^{\mathrm{T}} \boldsymbol{\phi}(s, a)}}
   {\sum_{b} e^{\boldsymbol{\theta}^{\mathrm{T}} \boldsymbol{\phi}(s, b)}}
\]
then:
\[
\pi(s, b)=\frac{e^{\boldsymbol{\theta}^{\mathrm{T}} \boldsymbol{\phi}(s, b)}}
   {\sum_{b} e^{\boldsymbol{\theta}^{\mathrm{T}} \boldsymbol{\phi}(s, b)}}
\]
I start by finding the derivative:
\begin{align}
  \label{der}
  \frac{\partial \pi(s, a)}{\partial \bs{\theta}} &=
  \frac{\bs{\phi}(s,a) e^{\boldsymbol{\theta}^{\mathrm{T}} \boldsymbol{\phi}(s, a)}}
       {\sum_{b} e^{\boldsymbol{\theta}^{\mathrm{T}} \boldsymbol{\phi}(s, b)}}
  -
  \frac{e^{\boldsymbol{\theta}^{\mathrm{T}} \boldsymbol{\phi}(s, a)}
        \sum_{b} \boldsymbol{\phi}(s, b)e^{\boldsymbol{\theta}^{\mathrm{T}} \boldsymbol{\phi}(s, b)}}
       {(\sum_{b} e^{\boldsymbol{\theta}^{\mathrm{T}} \boldsymbol{\phi}(s, b)})^2}\\
       &=\frac{e^{\boldsymbol{\theta}^{\mathrm{T}} \boldsymbol{\phi}(s, a)}}
              {\sum_{b} e^{\boldsymbol{\theta}^{\mathrm{T}} \boldsymbol{\phi}(s, b)}}
         \cdot
         \left(
         \boldsymbol{\phi}(s, a)-
         \frac{\sum_{b} \boldsymbol{\phi}(s, b)e^{\boldsymbol{\theta}^{\mathrm{T}} \boldsymbol{\phi}(s, b)}}
              {\sum_{b} e^{\boldsymbol{\theta}^{\mathrm{T}} \boldsymbol{\phi}(s, b)}}
         \right)
\end{align}
I can then use the compatibility condition to find the partial derivative:
\begin{align*}
  \frac{\partial f_{\boldsymbol{w}}(s, a)}{\partial \boldsymbol{w}}
  &=\frac{\partial \pi(s, a)}{\partial \boldsymbol{\theta}} \frac{1}{\pi(s, a)}\\
  &=\left(\frac{\sum_{b} e^{\boldsymbol{\theta}^{\mathrm{T}} \boldsymbol{\phi}(s, b)}}
               {e^{\boldsymbol{\theta}^{\mathrm{T}} \boldsymbol{\phi}(s, a)}}\right)
    \cdot
    \left(\frac{e^{\boldsymbol{\theta}^{\mathrm{T}} \boldsymbol{\phi}(s, a)}}
         {\sum_{b} e^{\boldsymbol{\theta}^{\mathrm{T}} \boldsymbol{\phi}(s, b)}}\right)
    \cdot
    \left(
    \boldsymbol{\phi}(s, a)-
    \frac{\sum_{b} \boldsymbol{\phi}(s, b)e^{\boldsymbol{\theta}^{\mathrm{T}} \boldsymbol{\phi}(s, b)}}
         {\sum_{b} e^{\boldsymbol{\theta}^{\mathrm{T}} \boldsymbol{\phi}(s, b)}}
    \right)\\
    &=
    \boldsymbol{\phi}(s, a)-
    \frac{\sum_{b} \boldsymbol{\phi}(s, b)e^{\boldsymbol{\theta}^{\mathrm{T}} \boldsymbol{\phi}(s, b)}}
         {\sum_{b} e^{\boldsymbol{\theta}^{\mathrm{T}} \boldsymbol{\phi}(s, b)}}\\
    &=
    \boldsymbol{\phi}(s, a)-
    \sum_{b} \frac{\boldsymbol{\phi}(s, b)e^{\boldsymbol{\theta}^{\mathrm{T}} \boldsymbol{\phi}(s, b)}}
         {\sum_{b} e^{\boldsymbol{\theta}^{\mathrm{T}} \boldsymbol{\phi}(s, b)}}\\
    &=
    \boldsymbol{\phi}(s, a)-
    \sum_{b} \frac{e^{\boldsymbol{\theta}^{\mathrm{T}} \boldsymbol{\phi}(s, b)}}
        {\sum_{b} e^{\boldsymbol{\theta}^{\mathrm{T}} \boldsymbol{\phi}(s, b)}
        }\boldsymbol{\phi}(s, b)\\
    &=
    \boldsymbol{\phi}(s, a)-
    \sum_{b}
    \pi(s, b)
    \boldsymbol{\phi}(s, b)
\end{align*}
\QEDA

\subsection*{Question 1.2}
\label{subsec:11}

\subsection*{Question 1.3}
\label{subsec:11}
\begin{align*}
  \frac{\partial}{\partial \boldsymbol{\theta}} \ln \pi(s, a) &=
  \frac{\sum_{b}(\bs{\phi}(s,a) - \bs{\phi}(s,b)) e^{\bs{\theta}^{T\bs{\phi}(s,b)}} }
       {\sum_{b}e^{\bs{\theta}^T\bs{\phi}(s,b)}}\\
 &=
  \frac{\sum_{b} e^{\boldsymbol{\theta}^{\mathrm{T}} \boldsymbol{\phi}(s, b)}}
       {e^{\boldsymbol{\theta}^{\mathrm{T}} \boldsymbol{\phi}(s, a)}}
  \left( \frac{e^{\boldsymbol{\theta}^{\mathrm{T}} \boldsymbol{\phi}(s, a)}
   \sum_{b}(\bs{\phi}(s,a) - \bs{\phi}(s,b)) e^{\bs{\theta}^{T\bs{\phi}(s,b)}} }
   {(\sum_{b}e^{\bs{\theta}^T\bs{\phi}(s,b)})^2}\right)\\
 &=
  \frac{1}{\pi(s,a)}
  \left(
  \frac{e^{\boldsymbol{\theta}^{\mathrm{T}} \boldsymbol{\phi}(s, a)}
    \sum_{b}(\bs{\phi}(s,a) - \bs{\phi}(s,b)) e^{\bs{\theta}^{T\bs{\phi}(s,b)}} }
       {(\sum_{b}e^{\bs{\theta}^T\bs{\phi}(s,b)})^2}
  \right)\\
 &=
  \frac{1}{\pi(s,a)}
  \left(
  \frac{e^{\boldsymbol{\theta}^{\mathrm{T}} \boldsymbol{\phi}(s, a)}
    \sum_{b}\bs{\phi}(s,a) e^{\bs{\theta}^{T\bs{\phi}(s,b)}} }
       {(\sum_{b}e^{\bs{\theta}^T\bs{\phi}(s,b)})^2}
  -
  \frac{e^{\boldsymbol{\theta}^{\mathrm{T}} \boldsymbol{\phi}(s, a)}
    \sum_{b}\bs{\phi}(s,b) e^{\bs{\theta}^{T\bs{\phi}(s,b)}} }
       {(\sum_{b}e^{\bs{\theta}^T\bs{\phi}(s,b)})^2}
  \right)\\
 &=
  \frac{1}{\pi(s,a)}
  \left(
  \frac{e^{\boldsymbol{\theta}^{\mathrm{T}} \boldsymbol{\phi}(s, a)}\bs{\phi}(s,a)}
       {\sum_{b}e^{\bs{\theta}^T\bs{\phi}(s,b)}}
  \left(
  \frac{\sum_be^{\boldsymbol{\theta}^{\mathrm{T}} \boldsymbol{\phi}(s, b)}}
       {\sum_{b}e^{\bs{\theta}^T\bs{\phi}(s,b)}}
  \right)
  -
  \frac{e^{\boldsymbol{\theta}^{\mathrm{T}} \boldsymbol{\phi}(s, a)}
    \sum_{b}\bs{\phi}(s,b) e^{\bs{\theta}^{T\bs{\phi}(s,b)}} }
       {(\sum_{b}e^{\bs{\theta}^T\bs{\phi}(s,b)})^2}
  \right)\\
 &=
  \frac{1}{\pi(s,a)}
  \left(
  \frac{e^{\boldsymbol{\theta}^{\mathrm{T}} \boldsymbol{\phi}(s, a)}\bs{\phi}(s,a)}
       {\sum_{b}e^{\bs{\theta}^T\bs{\phi}(s,b)}}
  -
  \frac{e^{\boldsymbol{\theta}^{\mathrm{T}} \boldsymbol{\phi}(s, a)}
    \sum_{b}\bs{\phi}(s,b) e^{\bs{\theta}^{T\bs{\phi}(s,b)}} }
       {(\sum_{b}e^{\bs{\theta}^T\bs{\phi}(s,b)})^2}
  \right)\\
&=
  \frac{1}{\pi(s,a)}
  \frac{\partial \pi(s, a)}{\partial \boldsymbol{\theta}} \tag{by (\ref{der})}\\
&=
  \frac{\partial \pi(s, a)}{\partial \boldsymbol{\theta}} 
  \frac{1}{\pi(s,a)}
\end{align*}
Which is the requirement of the \emph{compatibility condition}.\QEDA

\section{Offline Evaluation of Bandit Algorithms}
\label{sec:2}




\end{document}

%% \section{Empirical comparison of UCB1 and EXP3 algorithms}
%% \label{sec:5}
%% \begin{figure}
%%   \centering
%%   \begin{subfigure}[b]{0.49\textwidth}
%%     \centering
%%     \includegraphics[width=\textwidth]{code/plt_k2_mu025}
%%   \end{subfigure}
%%   \begin{subfigure}[b]{0.49\textwidth}
%%     \centering
%%     \includegraphics[width=\textwidth]{code/plt_k2_mu0375}
%%   \end{subfigure}
%%   \caption{Plots for Exercise 5}
%%   \label{plt5}
%% \end{figure}

%% \begin{lstlisting}[caption="Calculation of g"]
%% def calc_g(Xs, y, w):
%%     N = np.shape(Xs)[0]
%%     # use matrix X of xs instead of for-loop = much faster
%%     X = np.c_[Xs, np.ones(N)]
%%     num = y.T * X
%%     denum = 1 + np.exp(y * (w @ X.T))
%%     M = num.T/denum
%%     # return mean of each row
%%     return (-1 * np.mean(M, axis=1))
%% \end{lstlisting}
